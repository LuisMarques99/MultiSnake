\documentclass[\main.tex]{subfiles}

\chapter{Introdução}
\begin{singlespace}
\minitoc
\end{singlespace}
\vspace{20pt}

\section{Contextualização}
O presente documento visa descrever todo o trabalho realizado no desenvolvimento deste projeto,
que foi realizado para a prova oral da unidade curricular de \acrlong{lds} da
\acrlong{lei} na \acrlong{estg} do Instituto Politécnico Porto no ano de 2020.\par

\section{Âmbito}
Este projeto consiste no desenvolvimento de uma aplicação web para um jogo da cobra com a
vertente multijogador. Existe também um serviço de gestão de utilizadores com autenticação
\acrshort{jwt}. O utilizador terá a possibilidade de criar e de se juntar a uma sessão de jogo.
\par

\section{Objetivos}
Para a realização deste projeto, foram definidos alguns objetivos principais com a finalidade
de manter uma boa coesão e garantia de qualidade do produto e dos métodos de desenvolvimento
do mesmo. Alguns dos objetivos são obrigatórios da unidade curricular em que se insere o
trabalho e outros foram definidos pelo dono do produto e pelo gestor do projeto.\par
São eles:
\begin{itemize}
    \item Utilização da metodologia \gls{scrum} e uso da plataforma ;
    \item Uso do \gls{gitlab} e \gls{git} para gestão e versionamento do projeto, gestão de
    tarefas e \textit{wiki} do projeto;
    \item Desenvolvimento de serviços em \gls{aspnet} Core sob a forma de uma \acrfull{api} com autenticação
    de utilizadores via \acrfull{jwt};
    \item Documentação do projeto com ferramentas adequadas;
    \item Aplicação para o cliente capaz de consumir os serviços criados.
\end{itemize}

% Estrutura do documento

\newpage