\documentclass[\main.tex]{subfiles}

\chapter{Requisitos}
\begin{singlespace}
\minitoc
\end{singlespace}
\vspace{40pt}

Para a definição dos requisitos do projeto foram definidas três prioridades possíveis que podem
ser atribuidas a cada requisito e os respetivos pontos de realização do mesmo:
\begin{itemize}
    \item Alta: 8 pontos;
    \item Média: 5 pontos;
    \item Baixa: 2 pontos.
\end{itemize}\par
Estes pontos foram atribuidos com base na sequência de \textit{Fibonacci}. Analisando os pontos
associados a cada prioridade podemos ver que a realização de um requisito com prioridade alta tem
mais importância que a junção de um requisito de prioridade média com um de prioridade baixa.

\newpage

\section{Funcionais}
\subsection*{RF\_1.1}
\noindent\textbf{Nome:} O sistema tem de permitir o registo de um novo utilizador \par
% \noindent\textbf{User Story:} US\_1 \par
\noindent\noindent\textbf{Categoria:} Gestão de utilizadores \par
\noindent\textbf{Descrição:} O registo de um novo utilizador tem de ser realizado com o
preenchimento de alguns campos obrigatórios (email, nome de utilizador e palavra-passe). \par
\noindent\textbf{Prioridade:} Alta \par

\subsection*{RF\_1.2}
\noindent\textbf{Nome:} O sistema tem de permitir a atualização de um utilizador \par
\noindent\noindent\textbf{Categoria:} Gestão de utilizadores \par
\noindent\textbf{Descrição:} A atualização de um utilizador só é permitida para utilizadores já
registados no sistema e com o login realizado.\par
\noindent\textbf{Prioridade:} Média \par

\subsection*{RF\_1.3}
\noindent\textbf{Nome:} O sistema tem de permitir a atualização da palavra-passe de um
utilizador \par
\noindent\noindent\textbf{Categoria:} Gestão de utilizadores \par
\noindent\textbf{Descrição:} Para atualizar a palavra-passe de um utilizador tem de ser enviado
uma hiperligação para o email do utilizador em questão com uma validade de 5 minutos.\par
\noindent\textbf{Prioridade:} Média \par

\subsection*{RF\_1.4}
\noindent\textbf{Nome:} O sistema pode permitir a consulta da página de perfil de um
utilizador\par
\noindent\noindent\textbf{Categoria:} Gestão de utilizadores \par
\noindent\textbf{Descrição:} Qualquer utilizador sem estar autenticado pode consultar a página
de um utilizador em específico e ver as estatísticas dos jogos desse utilizador.\par
\noindent\textbf{Prioridade:} Baixa \par

\subsection*{RF\_2.1}
\noindent\textbf{Nome:} O sistema tem permitir a autenticação de um utilizador\par
\noindent\noindent\textbf{Categoria:} Gestão de sessões \par
\noindent\textbf{Descrição:} O login de um utilizador registado tem de ser suportado por uma
autenticação com \acrshort{jwt}.\par
\noindent\textbf{Prioridade:} Alta \par

\subsection*{RF\_2.2}
\noindent\textbf{Nome:} O sistema tem permitir o logout de um utilizador\par
\noindent\noindent\textbf{Categoria:} Gestão de sessões \par
\noindent\textbf{Descrição:} Um utilizador autenticado tem de ter a possibilidade de sair da
aplicação.\par
\noindent\textbf{Prioridade:} Alta \par

\subsection*{RF\_3.1}
\noindent\textbf{Nome:} O sistema deve permitir jogar o jogo da cobra na vertente multijogador\par
\noindent\noindent\textbf{Categoria:} Gestão o jogo \par
\noindent\textbf{Descrição:} O modo de jogo \textit{multiplayer} deve estar disponível para
qualquer utilizador autenticado.\par
\noindent\textbf{Prioridade:} Alta \par

\section{Não Funcionais}

\subsection*{RNF\_1.1}
\noindent\textbf{Nome:} A aplicação web tem de ser responsiva \par
\noindent\noindent\textbf{Categoria:} Aplicação web \par
\noindent\textbf{Descrição:} A aplicação web tem de ser o mais responsiva possível para que
possa ser utilizada a partir de qualquer dispositivo, visto que não irá existir aplicação
nativa.\par
\noindent\textbf{Prioridade:} Alta \par

\subsection*{RNF\_1.2}
\noindent\textbf{Nome:} A aplicação web tem de ter boa usabilidade \par
\noindent\noindent\textbf{Categoria:} Aplicação web \par
\noindent\textbf{Descrição:} A aplicação web tem de ser muito intuitiva e amiga do
utilizador, para que não seja difícil a adaptação à mesma.\par
\noindent\textbf{Prioridade:} Alta \par

\subsection*{RNF\_2.1}
\noindent\textbf{Nome:} A API tem de estar fortemente documentada \par
\noindent\noindent\textbf{Categoria:} API REST \par
\noindent\textbf{Descrição:} A API tem de estar fortemente documentada com a utilização do
\gls{swagger}.\par
\noindent\textbf{Prioridade:} Alta \par

\newpage

\subsection*{RNF\_3.1}
\noindent\textbf{Nome:} A \acrlong{bd} relacional tem de ser alojada num serviço cloud \par
\noindent\noindent\textbf{Categoria:} \acrlong{bd} \par
\noindent\textbf{Descrição:} A \acrlong{bd} relacional tem de ser alojada num serviço cloud da
\textit{Microsoft Azure}.\par
\noindent\textbf{Prioridade:} Alta \par

\newpage